\section{Background}
\label{sec:background}

\textbf{BFT.}
BFT protocols can tolerate up to $f$ node failures, where failed nodes can exhibit Byzantine behavior. 
% \tyler{node failures, where failed nodes can exhibit Byzantine behavior. Byzantine nodes can send arbitrary...} 
Byzantine nodes can send arbitrary messages and share private keys. They cannot, however, break standard cryptographic primitives~\cite{pbft}.
We assume a shared-nothing architecture in which nodes can only communicate through messages.
Messages between correct nodes will eventually be delivered.

\textbf{Rule-driven rewrites.}
\sigmodpaper{} presents a set of rule-driven rewrites that can be applied to arbitrary distributed protocols implemented in Dedalus~\cite{dedalus}, a declarative dataflow language for distributed systems based on Datalog.
Due to limited space, we will instead discuss these rewrites in the context of event-driven pseudocode instead.

\david{What the rewrites are.}
Rewrites improve protocol throughput by \emph{decoupling} and \emph{partitioning}~\cite{compartmentalized} (Figure~\ref{fig:rewrites}).
Decoupling splits \emph{logic} on a single node $l$ into two nodes $l_1$ and $l_2$.
Partitioning splits \emph{data} on a single node $l$ into multiple new nodes $\overline{l} = \{l_1, l_2, \ldots\}$.
Both rewrites alleviate single-node bottlenecks by introducing pipeline- and data- parallelism, respectively~\cite{autocomp}.
We say that these new nodes $l_1, l_2, \ldots$ \textbf{correspond} to $l$.

\david{Vulnerability in the rewrites.}
Used naively, decoupling and partitioning can make a protocol that was previously fault tolerant vulnerable to Byzantine attacks.
The rewrites can (1) create new nodes and introduce message channels between them, in order to enable pipelining or coordination, and (2) modify existing message channels so messages to a single node must be forwarded to its correct corresponding nodes.
% \natacha{Remind the reader why that's a problem?}.
New message channels present exploitable opportunities to Byzantine nodes, which can send arbitrary messages to those channels.
% \natacha{given below you have to add predicates like isByzantine, it would be super helpful to have a mini pseudocode example here to make the rest of the section easier to undertand. I would say that this is more important (to me) than the figure with the CRDT}


\subsection{Running example and syntax}
\label{sec:syntax}
We will use \Cref{alg:running-example} as a case study in our rewrites.
% \heidi{but our approach could equally be applied to example x, Y and Z}

\david{Introduce running example}
This snippet of PBFT~\cite{pbft} collects \textsc{Prepare} messages, waits for a quorum, then broadcasts a \textsc{Commit} message, representing a common pattern that can be found in many other BFT protocols.
\textsc{Prepare} messages arrive on the \texttt{prepareIn} channel and \textsc{Commit} messages are sent on the \texttt{commitOut} channel.
We will use the syntax \texttt{commitOut@dest $\leftsquigarrow m$} to indicate sending a message $m$ to node \texttt{dest} on channel \texttt{commitOut}.
The address of each node is stored in the variable \texttt{self}, and the addresses of all nodes is stored in \texttt{allNodes}.

Each \textsc{prepare} message has the following fields: view $v$, slot number $n$, digest $d$ of a command, sender index $i$, and signature $\sigma$.
Each \textsc{Prepare} message is first filtered based on its signature and view (\Cref{alg:running-example-verify}), then added to the log (\Cref{alg:running-example-add-to-log}).
Once $2f+1$ \textsc{Prepare} messages have been received for a particular slot number and digest (\Cref{alg:running-example-quorum}), a \textsc{Commit} message is signed and broadcasted to all other nodes (\Cref{alg:running-example-sign,alg:running-example-broadcast}).

\david{Signatures.}
Nodes have access to the cryptographic functions \texttt{sign(sk, m)} and \texttt{verify(pk, $\sigma$, m)}.
\texttt{sign(sk, m)} returns the signature $\sigma$ for a secret key $sk$ and message $m$.
\texttt{verify(pk, $\sigma$, m)} returns true if $\sigma$ is the signature of the secret key $sk$ corresponding to the public key $pk$ over $m$; it returns false otherwise.
Keys are available to each node through the maps \texttt{skeys} and \texttt{pkeys}; for each node at location $l$, \texttt{skeys[$l'$]} and \texttt{pkeys[$l'$]} returns the secret and public keys used to sign and verify messages from $l$ to $l'$, respectively.

% \david{Simplifications from PBFT.}
% For simplicity, this snippet differs from PBFT in the following ways:
% (1) A quorum is $2f+1$ \textsc{Prepare}s instead of $2f$ \textsc{Prepare}s and 1 \textsc{Pre-Prepare}.
% \natacha{please don't mention this. You're just going to confuse people and you're not changing anything}
% (2) The signature consists of a single MAC for a specific receiver to verify, as opposed to an array of MACs (an \emph{authenticator}~\cite{pbft}) that can be verified by any node.
% \natacha{Are you talking about macs or signatures, they are not the same thing. You have a single signature or n macs. What  are you signing?}

\david{Equivalence between MACs and signatures.}
We treat the signature schemes commonly used by BFT protocols---MACs (Message Authentication Codes)~\cite{MAC} and public-key signatures~\cite{signature}---as black boxes and do not distinguish between them in our formalism.
Both produce a signature that can be verified with the correct key; this key is symmetric for MACs and asymmetric for public-key signatures.

\david{Make clear which keys are available to which nodes for MACs and public-key signatures.}
In our \jmh{"discussion" rather than "formalism"? it doesn't seem very "formal".} formalism, we will store all keys, symmetric or asymmetric, in \texttt{skeys} and \texttt{pkeys}, and sign and verify them with the same syntax.
Any symmetric key $k$ used between nodes $l$ and $l'$ is \emph{only} available to those two nodes: $k=$ \texttt{skeys[$l'$]} $=$ \texttt{pkeys[$l'$]} on $l$, and $k=$ \texttt{skeys[$l$]} $=$ \texttt{pkeys[$l$]} on $l'$.
The availability of asymmetric keys is asymmetric.
A node $l$ signs all its outgoing messages with $sk$, which is known to no other node, but all nodes know its public key $pk$: for any $l'$, $sk=$ \texttt{skeys[$l'$]} on $l$, and on any $l'$, $pk=$ \texttt{pkeys[$l$]}.
% (true if \texttt{sk = pk} for MACs, and true if \texttt{sk} is the secret key corresponding to \texttt{pk} for public-key signatures)


\begin{algorithm}
\caption{The running example from PBFT.}
\label{alg:running-example}
prepareLog = \{\}\;
numPrepares[][] = \{\}\;

\For{msg $\in$ prepareIn(v, n, d, i, $\sigma$)}{
    \If{verify(pkeys[i], $\sigma$, <v, n, d>) \And v = view}{ \label{alg:running-example-verify}
        prepareLog.add(msg)\; \label{alg:running-example-add-to-log}
        numPrepares[n][d] += 1\;

        \If{numPrepares[n][d] $\ge 2f+1$}{ \label{alg:running-example-quorum}
            \For{dest $\in$ allNodes}{
                $\sigma' \gets$ sign(skeys[dest], <v, n, d, self>)\; \label{alg:running-example-sign}
                commitOut@dest $\leftsquigarrow$ (v, n, d, self, $\sigma'$) \label{alg:running-example-broadcast}
            }
        }
    }
}
\end{algorithm}

% \heidi{You can save space in all the pseudo code by removing the "end"s. The indentation is enough to make the flow control clear (python style)}


\subsection{Correctness}
\label{sec:correctness}

This paper is focused on proving the correctness of \emph{rewrites} across BFT protocols, not the correctness of any specific BFT protocol.
Assuming that the original distributed protocol is correct in the presence of $f$ Byzantine failures, the modified rewrites will decouple and partition the protocol while preserving correctness up to $f$ Byzantine failures.
% \heidi{I think you want to be even more explicit here, we assume that the original distributed protocol is correct in the presence of f byz failures. Our transformations will decouple and partition the protocol across more nodes to improve performance whilst preserving the correctness in the presence of f faults.}

\david{Fault domains.}
We discuss failures in the rewritten protocol in terms of \emph{fault domains}.
Each node $l$ in the original protocol belongs to its own fault domain; when $l$ is scaled up into its corresponding nodes $\overline{l}$, all nodes in $\overline{l}$ remain in the same fault domain.
Therefore, the failure of \emph{any} node in $\overline{l}$ is equivalent \jmh{rather than "is equivalent" shouldn't this be "must appear to an observer to be equivalent"} to the failures of \emph{all} nodes in $\overline{l}$, which is equivalent to the failure $l$, and the rewrites do not change the number of fault domains.

Intuitively, to any observer, the output of all nodes $\overline{l}$ together represents the output of the corresponding original node $l$.
A ``partially'' Byzantine node is still Byzantine: if any single node $l_i \in \overline{l}$ becomes Byzantine while the remaining corresponding nodes are still correct, it is as if the original node $l$ becomes Byzantine but continues to send \textit{some} correct messages.
% \david{Joe and Natacha: Worth a sentence justifying fault domains because of the cloud, nodes are managed by entities.}

\david{Definition of correctness.}
Correctness is defined by observable program behavior.
A rewrite is correct if given any program $P$, a rewritten program $P'$, and any set of inputs (and their respective send times), $P'$ always generates the same outputs with the same timestamps as some possible run of $P$ with up to $f$ Byzantine failures~\cite{autocomp}.
% \nc{Is it worth saying again that in this case, a possible run of P includes byz behaviour from up to f machines?}


