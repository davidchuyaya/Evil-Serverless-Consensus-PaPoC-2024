\section{Modifications to rewrites}
\label{sec:modifications-to-rewrites}

We now present two modifications to the rewrites introduced by \sigmodpaper{}: sender verification (\Cref{sec:sender-verification}) and message verification (\Cref{sec:message-verification}).

\subsection{Sender verification}
\label{sec:sender-verification}
% \jmh{Let's use parallel language to 4.2: "The rewrites introduce new channels, with implicit assumptions about which nodes will send on those channels."}
The rewrites introduce new channels, with implicit assumptions about which nodes will send on those channels.\footnote{Rewrites that fall into this category include monotonic decoupling, functional decoupling, asymmetric decoupling, and partial partitioning~\cite{autocomp,autocompTechReport}.}
% \heidi{this sentence might be good as a footnote. atm it might not contribute anything to readers who are unfamiliar with the other paper}
Without additional verification, these channels would be vulnerable to Byzantine attacks from other nodes.

\textbf{Modification: Sender verification.}
Replace each new channel \texttt{chan(m)} from node $l_1$ to $l_2$ with \texttt{signed\_chan(m, $\sigma$)}, where $\sigma =$ \texttt{sign(skeys[$l_2$], m)}.
The receiving node $l_2$ only processes a message $m$ if \texttt{verify(pkeys[$l_1$], $\sigma$, m)} returns true.

\jmh{This feels like a minor point -- could be cut if space is an issue. The only thing I like about it is that it captures what PBFT is doing in a more general way.}
As an optimization, if functional dependencies~\cite{autocomp,alice} are known, then the sender does not have to sign all fields of the message.
If there is a functional dependency from message field $x$ to $y$---that is, there is some deterministic function $f$ where $f(x) = y$---then signing over $x$ is sufficient.
If there is a one-to-one dependency between the fields---there exists some $f^{-1}(y) = x$---then signing over \emph{either} field is sufficient.
This optimization applies to any BFT protocol that uses collision-resistant hash functions (such as PBFT), where there is a one-to-one dependency between the hash digest and the original message.
In that case, only the digest has to be signed.
PBFT itself makes the same observation and only signs the digest on its preexisting channels.
Note that the inverse of the hash function does not need to be known; this optimization is valid as long as such a function \emph{exists}.

% , adding \Cref{alg:sender-verification-sender} to the sender and \Cref{alg:sender-verification-receiver} to the receiver.
% We create a new set of keys $sk, pk$ to be shared between the sender and receiver.
% \david{Is the pseudocode beneath necessary? They're very simple.} \heidi{No}

% \begin{algorithm}
% \caption{Modifications to the sender for sender verification for each message $m$ originally sent on \texttt{chan} to node $l_2$.}
% \label{alg:sender-verification-sender}
% $\sigma \gets$ sign$(sk, m)$\;
% signed\_chan@$l_2 \leftsquigarrow$ (m, $\sigma$)
% \end{algorithm}

% \begin{algorithm}
% \caption{Modifications to the receiver for sender verification.}
% \label{alg:sender-verification-receiver}
% \For{(m, $\sigma$) $\in$ signed\_chan(m, $\sigma$)}{
%     \If{verify(pk, $\sigma$, m)}{
%         chan(m) $\gets$ m
%     }
% }
% \end{algorithm}


\subsection{Message verification}
\label{sec:message-verification}

The partitioning rewrites also introduce channels with implicit assumptions on message content.
Nodes created through partitioning all share the same types of input channels, but each partition assumes that it will receive a specific disjoint subset of the messages.\footnote{All subcategories of partitioning in~\cite{autocomp}---partitioning with co-hashing, partitioning with dependencies, and partial partitioning---share this assumption.}
Byzantine nodes can exploit this assumption and send messages to the wrong nodes.

\textbf{Modification: Message verification.}
\jmh{perhaps index with the partition index $i$? e.g. \texttt{correct\_chan(m, i)}? Or add a filter function \texttt{destination\_filter(chan(m), i)}?}
For each input channel \texttt{chan(m)} on partition $l_i$, replace all instances of \texttt{chan(m)} with \texttt{correct\_chan(m)}, which drops any messages from \texttt{chan(m)} that are not meant for this partition.
This criteria can be checked with the distribution policy function $D(m)$ as defined in~\cite{autocomp}, which maps each message to its intended partition.

% , as seen in \Cref{alg:message-verification}.
% The function $D(m)$ is the distribution policy that determines the routing of messages to partitions~\cite{autocomp}.

% \begin{algorithm}
% \caption{Modifications to the partition for message verification.}
% \label{alg:message-verification}
% \For{m $\in$ chan(m)}{
%     \If{$D(m) = i$}{
%         correct\_chan(m) $\gets$ m
%     }
% }
% \end{algorithm}

% Partitioning splits data on a single original machine $l$ onto multiple machines $l_1, l_2, \ldots$ with the same logic.
% After partitioning, any input channel on $l$ will become an input channel on \emph{all} partitions $l_1, l_2, \ldots$, and any machine that sent messages to $l$ would have to reroute the message to the correct partition, depending on the output of the partitioning function $D(m)$ on the message $m$.
% A Byzantine machine can misbehave by rerouting or duplicating messages to the wrong partition(s). \david{Reference example in intro?}
% Because the rewrites were introduced assuming no Byzantine failures, partitions did not check whether the messages they received were redirected correctly.
% We introduce the following modification to the partitioning rewrites to verify that incorrectly routed messages are discarded.

% \textbf{Rewrite: Partitioning verification.}
% For each input channel \texttt{in(...)} on partition $l_i$ with index $i$, we replace all instances of \texttt{in(...)} in $l_i$ with \texttt{in\_correct(...)} and add the following logic:
% \begin{algorithm}
% \caption{Partitioning verification rewrites}
% \For{m $\in$ in(...)}{
%     \If{$D(m) = i$}{
%         in\_correct $\gets$ m
%     }
% }
% \end{algorithm}