\begin{abstract}

% \heidi{Nit pick: I prefer BFT protocols instead of Byzantine protocols}

Byzantine Fault Tolerant (BFT) protocols provide powerful guarantees in the presence of arbitrary machine failures, yet they do not scale.
% \heidi{I love BFT but this might be overclaiming, bft protocols are mostly used in blockchains. maybe mention that they are powerful or something else.}
The process of creating new, scalable BFT protocols requires expert analysis and is often error-prone.
Recent work suggests that localized, rule-driven rewrites can be mechanically applied to scale existing (non-BFT) protocols, including Paxos.
We modify these rewrites---decoupling and partitioning---so they can be safely applied to BFT protocols, and apply these rewrites to the critical path of PBFT, improving its throughput by $5\times$.
We prove the correctness of the modified rewrites on \emph{any} BFT protocol by formally modeling the arbitrary logic of a Byzantine node.
We define the \randomSimulator{}, a theoretical node that simulates a Byzantine node through randomness, and show that in any BFT protocol, the messages that a \randomSimulator{} can generate before and after optimization is the same.
Our initial results point the way towards an automatic optimizer for BFT protocols.

\end{abstract}